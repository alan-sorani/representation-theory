\documentclass[10pt,a4paper,twoside,openany,hidelinks]{book}

%JAPANESE
\usepackage{fontspec}
\defaultfontfeatures{Ligatures={NoCommon, NoDiscretionary, NoHistoric, NoRequired, NoContextual}}
\setmainfont{CMU Serif}
\usepackage{xeCJK}
\xeCJKsetup{CJKmath=true}
\setCJKmainfont{MS Mincho} % or some other Japanese font

\usepackage{maths}
\usepackage{stylish}

\usepackage{tkz-graph}
\GraphInit[vstyle = Shade]

\title{Lecture Notes to Fundamental Concepts in Representation Theory \\ \large{Spring 2020, Hebrew University of Jerusalem}}
\author{Alexander Yom Din\\ \large{Typed by Elad Tzorani}}
\date{\today}

\usepackage{lipsum}
\usepackage{stmaryrd}

\tikzset{EdgeStyle/.style   = {thick,
                               double          = orange,
                               double distance = 1pt}}

\tikzset{
    labl0/.style={anchor=south, rotate=-30, inner sep=.5mm}
}

\tikzset{
    labl1/.style={anchor=south, rotate=30, inner sep=.5mm}
}

\tikzset{
    labl2/.style={anchor=south, rotate=60, inner sep=.5mm}
}

\tikzset{
    labl3/.style={anchor=south, rotate=90, inner sep=.5mm}
}

\tikzset{
    labl4/.style={anchor=south, rotate=-90, inner sep=.5mm}
}

\usepackage{scalerel,stackengine}
\stackMath
\newcommand\widecheck[1]{%
\savestack{\tmpbox}{\stretchto{%
  \scaleto{%
    \scalerel*[\widthof{\ensuremath{#1}}]{\kern-.6pt\bigwedge\kern-.6pt}%
    {\rule[-\textheight/2]{1ex}{\textheight}}%WIDTH-LIMITED BIG WEDGE
  }{\textheight}% 
}{0.5ex}}%
\stackon[1pt]{#1}{\scalebox{-1}{\tmpbox}}%
}
\parskip 1ex

\newcommand{\from}{\leftarrow}

\usepackage[nottoc]{tocbibind}

\begin{document}
\frontmatter
\frontpage{front}{0.8\textwidth}{}
\tableofcontents
\countlectures
\newpage

\chapter*{Preface}
\addcontentsline{toc}{chapter}{Preface} \markboth{Preface}{}

\section*{Technicalities}
\addcontentsline{toc}{section}{Technicalities} %\markboth{Technicalities}{}

These aren't formal notes related to the course and henceforward there is \emph{absolutely no guarantee} that the recorded material is in correspondence with the course expectations, or that these notes lack any mistakes.\\
In fact, there probably are mistakes in the notes! I would highly appreciate if any comments or corrections were sent to me via email at \href{mailto:tzorani.elad@gmail.com}{tzorani.elad@gmail.com}.\\
Elad Tzorani.

\mainmatter

\chapter*{Overview}

This course will first discuss representation theory of finite groups and semisimple modules and rings.
These are somehow related as described in the notes.

We later discuss character theory, induction and related topics.

If time permits, follows an introduction to representation theory of compact groups.

\chapter{$G$-Groups and $G$-Sets}

\begin{notation}
Let $X \in \Set$, we denote by $S_X$ the group of set automorphisms of $X$.
\end{notation}

\begin{definition}[$G$-Set]
Let $G \in \Grp$. A \stress{$G$-set} is a set $X$ equipped with a group homomorphism $\rho \colon G \to S_X$.
\end{definition}

\begin{remark}
Equivalently a $G$-set is a set $X$ with a map $a \colon G \times X \to X$ such that
\begin{align*}
a\prs{1, x} &= x \\
a\prs{g_1 g_2, x} &= a\prs{g_1, a\prs{g_2, x}} \text{.}
\end{align*}

This is given by
\[\rho\prs{g}\prs{x} = a\prs{g, x} \text{.}\]
\end{remark}

\begin{notation}
We don't usually write $\rho, a$ explicitly. We instead write $gx \ceq \rho\prs{g}\prs{x}$.
\end{notation}

\newcommand{\acts}{\curvearrowright}

\begin{example}
Let $V \in \Vect_\RR^{\fin}$ equipped with an inner product.

Let $\O\prs{V} \subseteq \G\L\prs{V}$ be the orthogonal group and let
\[S\prs{V} \ceq \set{v \in V}{\norm{v} = 1} \text{.}\]
Then $\O\prs{V}$ acts on $S\prs{V}$, which we denote
$O\prs{V} \acts S\prs{V}$.
\end{example}

\begin{example}[Action on Quotient]
Let $G \in \Grp$ and $H \leq G$ a subgroup. Then
$G \acts \quot{G}{H}$ by
\[g \cdot \tilde{g} H = g \tilde{g} H \text{.}\]
\end{example}

\begin{definition}[Morphism of $G$-Sets]
A \stress{morphism} of $G$-Sets $X, Y$ is a map $f \colon X \to Y$ such that
\[\forall g \in G \forall x \in X \colon f\prs{gx} = g f\prs{x} \text{.}\]
\end{definition}

\begin{definition}[Isomorphism of $G$-Sets]
An \stress{isomorphism} of $G$-sets is a morphism which admits an inverse morphism.
\end{definition}

\begin{exercise}
A bijective morphism of $G$-sets is an isomorphism.
\end{exercise}

\newcommand{\Maps}{\mrm{Maps}}

\begin{notation}
We denote the set of morphisms from $G$-sets $X$ to $Y$ by $\Maps_G\prs{X,Y}$.
\end{notation}

\begin{definition}[Transitive $G$-Set]
A $G$-set $X$ is \stress{transitive} if it's nonempty and
\[\forall x_0, x_1 \in X \exists g \in G \colon g x_0 = x_1 \text{.}\]
\end{definition}

\begin{notation}
Let $G \in \Grp$, we denote the category of $G$-sets with morphisms of $G$-sets by $G\text{-}\Set$.
\end{notation}

\begin{definition}
Let $X \in G\text{-}\Set$ and let $x_0 \in X$.
Denote the \stress{stabiliser}
\[\mrm{Stab}_G\prs{x_0} = \set{g \in G}{g x_0 = x_0} \subseteq G\]
which is a subgroup of $G$.
\end{definition}

\begin{remark}
\[\quot{G}{\mrm{Stab}_G\prs{x_0}} \cong X \text{.}\]
\end{remark}

\begin{example}
Let $e = \pmat{0 \\ \vdots \\ 0 \\ 1} \in V$ and let $W = \spn\prs{e}^\perp$.
Then
\[\mrm{Stab}_{\O\prs{V}} \prs{e} \cong \O\prs{W} \text{.}\]
\end{example}

\begin{example}
Let $X \in \Set$, there's a natural action $S_X \acts X$.
\end{example}

\begin{example}
$S_2 \ceq S_{\set{1,2}} \acts \set{1,2}$ is indecomposable in some sense. We cannot decompose the permutation $\prs{1 2}$.

We have however $\prs{1 2}^2 = \id$.
In the case of basis permutation in linear algebra, in such cases we'd like to decompose the matrix $\pmat{0 & 1 \\ 1 & 0}$ to its restriction to eigenspaces with eigenvalues $\pm 1$.
\end{example}

\begin{notation}
Let henceforth $k \in \catname{Field}$, $G \in \Grp$.
\end{notation}

\begin{definition}[Group Representation]
A \stress{representation of $G$ over $k$} is $V \in \Vect_k$ together with a group homomorphism
\[\rho \colon G \to \GL\prs{V} \text{.}\]
\end{definition}

\begin{notation}
We denote as above $gv \ceq \rho\prs{g}\prs{v}$.
\end{notation}

\begin{definition}[Morphism of $G$-Representations]
A \stress{morphism of $G$-representations from $V$ to $W$} is a linear map $T \colon V \to W$ such that
\[\forall v \in V \forall g \in G \colon T\prs{gv} = g T\prs{v} \text{.}\]
\end{definition}

\begin{notation}
Denote by $\hom_G\prs{V,W}$ the set of $G$-representation-morphisms from $V$ to $W$.
\end{notation}

\begin{remark}
$\hom_G\prs{V,W} \subseteq \hom\prs{V,W}$ is a $k$-sub-vector-space.
\end{remark}

\begin{definition}[Isomorphism of $G$-Representations]
A \stress{isomorphism of $G$-representations from $V$ to $W$} is a morphism which admits an inverse.
\end{definition}

\begin{exercise}
An bijective morphism of $G$-representations is an isomorphism.
\end{exercise}

\begin{notation}
Denote $S_n \ceq S_{\set{1, \ldots, n}}$.
\end{notation}

\begin{notation}
Let $X \in \Set$, we denote by $\Fun_k\prs{X}$ the vector space of functions $X \to k$.
\end{notation}

\begin{example}
$S_n \acts k^n$ via
\[\sigma\prs{x_1 \\ \vdots \\ x_n} = \pmat{x_{\sigma^{-1}\prs{1}} \\ \vdots \\ x_{\sigma^{-1}\prs{n}}} \text{.}\]

More generally
$\Fun_k\prs{X}$ has an action $S_X \acts \Fun_k\prs{X}$ via
\[\prs{\sigma f}\prs{x} \ceq f\prs{\sigma^{-1}\prs{x}} \text{.}\]
\end{example}

\begin{notation}
We denote by $k\brs{X}$ the vector space spanned by the symbols $\set{\delta_x}{x \in X}$.
\end{notation}

\begin{remark}
If $X$ is finite, there's a natural isomorphism $\Fun_k\prs{X} \cong k\brs{X}$.
\end{remark}

\begin{remark}
There's an action $S_X \acts k\brs{X}$ via
\[\forall \sigma \in S_X \forall x \in X \colon \sigma \delta_x = \delta_{\sigma\prs{x}} \text{.}\]
\end{remark}

\begin{remark}
Given a map $X \to Y$ we have induced maps
\begin{align*}
k\brs{X} &\to k\brs{Y} \\
\Fun_k\prs{Y} &\to \Fun_k\prs{X} \text{.}
\end{align*}
\end{remark}

\begin{example}[Trivial Representations]
Let $c \in k$, there's an action $G \acts k$ via $g \cdot c = c$.
\end{example}

\begin{definition}[Character]
A \stress{character} is a group homomorphism $G \to k^\times$.
\end{definition}

\begin{notation}
Given a character $\chi \colon G \to k^\times$ we consider the $G$-representation $k_\chi$, which as a vector space is $k$, and the homomorphism being
$g \cdot c \ceq \chi\prs{g} \cdot c$.
\end{notation}

\begin{exercise}
Show that \[\chi \mapsto k_\chi\]
givens a bijection from the set of characters $G \to k^\times$ to the set of isomorphism classes of $1$-dimensional $G$-representations over $k$.
\end{exercise}

\begin{definition}[$G$-Subrepresentation]
Let $V$ a $G$-representation and $W \leq V$ a subspace. We say $W$ is a \stress{$G$-subrepresentation of $V$} if
\[\forall g \in G \forall w \in W \colon gw \in W \text{.}\]
\end{definition}

\begin{example}
Let $S_n \acts k^n$, we can consider the subrepresentation
\[W \ceq \set{\pmat{x_1 \\ \vdots \\ x_n}}{x_1 + \ldots + x_n = 0} \subseteq k^n\]
where the condition $x_1 + \ldots + x_n = 0$ is indeed invariant under permutations.
\end{example}

\begin{definition}[Quotient Representation]
Let $V$ a $G$ representation and $W \subseteq V$ a $G$-subrepresentation.
We define \stress{the quotient representation} $\quot{V}{W}$ via
\[g\prs{v + W} = gv + W \text{.}\]
\end{definition}

\begin{definition}[Irreducible (alt. Simple) Representation]
We say that a $G$-representation $V$ is \stress{irreducible} if $V \neq 0$ and there are no subresentations of $V$ except $0$ and $V$.
\end{definition}

\begin{definition}[Indecomposable Representation]
We say that a $G$-representation $V$ is \stress{indecomposable} if given $W_1, W_1 \leq V$ such that $V = W_1 \oplus W_2$ we have \[W_1 = 0 \, \vee \, W_2 = 0 \text{.}\]
\end{definition}

\begin{example}
Let $S_2 \acts k^2$ as before.
Assume that $\mrm{char}\prs{k} = 2$.
Then $k^2$ is not irreducible. It has the subrepresentation \[\set{\pmat{x_1 \\ x_2}}{x_1 + x_2 = 0} \text{.}\]

However, $k^2$ is indecomposable.
Informally, if $k^2 = L_1 \oplus L_2$ is nontrivial, $L_1, L_2$ are one-dimensional so are isomorphic to the characters %TODO set? group?
Then
\[\prs{\chi\prs{1 2}}^2 = 1 \implies \chi\prs{1 2} = 1\]
so $L_1, L_2 \cong k$ so $\prs{1 2}$ should act as the identity on $k^2$, a contradiction.
\end{example}

\begin{notation}
We sometimes say $G$-morphism to mean a morphism of $G$-representations.
\end{notation}

\begin{definition}[Kernel]
Let $T \colon V \to W$ a $G$-morphism, we define the \stress{kernel of $T$} to be
\[\ker\prs{T} = \set{v \in V}{T\prs{v} = 0} \text{.}\]
\end{definition}

\begin{definition}[Direct Sum]
Let $V,W$ be $G$-representations.
We define $V \oplus W$ to be $V \times W$ as a set, with the action
\[g \prs{v,w} = \prs{gv, gw} \text{.}\]
\end{definition}

\begin{definition}[Semisimple Representation]
Let $V$ a $G$-representation, we say $V$ is \stress{semisimple} if for any subrepresentation $W \leq V$ there exists a subrepresentation $U \leq V$ such that
\[V = W \oplus U \text{.}\]
\end{definition}

\begin{remark}
A representation which is semisimple and indecomposable is irreducible.
\end{remark}

\begin{theorem}[Maschke] \label{theorem:maschke}
Let $G \in \Grp^\fin$ and assume $\mrm{char}\prs{k} \nmid \abs{G}$.

Then every finite-dimensional $G$-representation over $k$ is semisimple.
\end{theorem}

\begin{proof}
Let $V$ be a finite-dimensional $G$-representation over $k$ and let $\rho \colon G \to \GL\prs{V}$ be the corresponding homomorphism.
Let $W \leq V$ a subrepresentation.

Let $P \colon V \to V$ be a projection operator onto $W$.
I.e.
\begin{align*}
\rest{P}{W} = \id_W, \quad \im\prs{P} = W \text{.}
\end{align*}
Let
\begin{align*}
Q \ceq \frac{1}{\abs{G}} \sum_{g \in G} \rho\prs{g} \circ P \circ \rho\prs{g}^{-1} \in \endo\prs{V} \text{.}
\end{align*}

We want to show $Q$ is a projection on $W$.
Indeed
\begin{itemize}
\item
We have $\rho\prs{g}\prs{W} \subseteq W$ for every $g$, hence $\im Q \subseteq W$.
\item
If $w \in W$ we have
\[\rho\prs{g} P \rho\prs{g}^{-1} w = \rho\prs{g} \rho\prs{g}^{-1} \prs{w} = w\]
where in the first equality we use the fact that $\rest{P}{W} = \id_W$.
Hence $Q w = w$ so $\rest{Q}{W} = \id_W$.
\end{itemize}

We want to show $Q$ is a $G$-morphism.
I.e. that for every $h \in G$ we have $Q \circ \rho\prs{h} = \rho\prs{h} \circ Q$.
Indeed
\begin{align*}
Q \circ \rho\prs{h} &= \frac{1}{\abs{G}} \sum_{g \in G} \rho\prs{g} \circ P \circ \rho\prs{g}^{-1} \circ \rho\prs{h} \\&=
\frac{1}{\abs{G}} \sum_{g \in G} \rho\prs{g} \circ P \circ \rho\prs{\prs{h^{-1} g}^{-1}}
\\&\stackrel[g \mapsto hg]{}{=}
\frac{1}{\abs{G}} \sum_{g \in G} \rho\prs{hg} \circ P \circ \rho\prs{g^{-1}}
\\&=
\rho\prs{h} \circ \prs{\frac{1}{\abs{G}} \sum_{g \in G} \rho\prs{g} \circ P \circ \rho\prs{g}^{-1}}
\\&=
\rho\prs{h} \circ Q \text{.}
\end{align*}

So, $Q$ is a $G$ morphisms so $V = W \oplus \ker Q$, as required.
\end{proof}

\begin{definition}
Let $V,W$ be $G$-representations. We define $G \acts \hom\prs{V,W}$ via
\[\forall g \in G \forall T \in \Hom\prs{V,W} \forall v \in V \colon \prs{gT}\prs{v} = g T \prs{g^{-1} v} \text{.}\]
\end{definition}

\begin{definition}[Subspace of Invariants]
Let $V$ be a $G$-representation. We define the \stress{subspace of invariants}
\[V^G \ceq \set{v \in V}{\forall g \in G \colon gv = v} \subseteq V \text{.}\]
\end{definition}

\begin{definition}[Averaging Operator]
Let $V$ be a $G$-representation and assume $G \in \Grp^\fin$. We define the \stress{averaging operator}
\[\mrm{Av}_V^G \prs{v} \ceq \frac{1}{\abs{G}} \sum_{g \in G} gv \text{.}\]
\end{definition}

\begin{remark}
$\mrm{Av}_V^G$ is a $G$-morphism and a projection operator with image $V^G$.
\end{remark}

Assume throughout that $G$ is finite, $V,W$ are $G$-representations and $\mrm{char}\prs{k} \nmid \abs{G}$.

\begin{exercise}
Prove that
\[\hom_G\prs{V,W} = \hom\prs{V,W}^G\]
where the right-hand side is the set of fixed points under the $G$-action.
\end{exercise}

\begin{proof}[2\textsuperscript{nd} proof for \ref{theorem:maschke} where $k \in \set{\RR,\CC}$]

Assume $V$ is finite-dimensional and $W \leq V$.
Let $\trs{\cdot,\cdot}_0$ be an inner product on $V$.

Denote
\[\trs{v_1, v_2} \ceq \frac{1}{\abs{G}} \sum_{g \in G} \trs{g v_1, g v_2}_0 \text{.}\]
This is then an inner product and is $g$-invariant in the sense that $\trs{g v_1, g v_2} = \trs{v_1, v_2}$.

Let $U$ be the orthogonal complement of $W$ with respect to $\trs{\cdot , \cdot}$.
Then $V = W \oplus U$ and we're left to check that $U$ is $G$-invariant (then a $G$-representation).

Indeed
\[\trs{w, gu} = \trs{g^{-1}w, g^{-1} g u} = \trs{g^{-1}w, u} = 0 \text{.}\]
\end{proof}

\begin{proof}[3\textsuperscript{rd} proof for \ref{theorem:maschke}]
Consider the projection
\[P \colon V \repi \quot{V}{W} \text{.}\]
We want to find a $G$-morphism $i \colon \quot{V}{W} \to V$ such that $P \circ i = \id$.
Then $\im\prs{i} \oplus W = V$ by the \href{https://ncatlab.org/nlab/show/split+exact+sequence}{splitting lemma}.

More generally, given $G$-representations $X,Y,Z$ and a $G$-morphism $\phi \colon Y \to Z$ we want to see that
\[\hom_G\prs{X,Y} \xrightarrow{\phi \circ -} \hom_G\prs{X,Z}\]
is surjective.
In our case $Z = \quot{V}{W}$ and $X \ceq \quot{V}{W}$ so $\id \in \hom_G\prs{\quot{V}{W}, \quot{V}{W}}$ has a preimage in $\hom_G\prs{\quot{V}{W}, V}$ under $P \circ -$.

Consider
\[\hom\prs{X, Y} \xrightarrow{P \circ -} \hom\prs{X, Z} \text{.}\]
This is a $G$-morphism and is surjective by linear algebra.

Reformulating the problem, given $G$-representations $V,Z$ and a surjective morphism $p \colon V \repi Z$ the resulting map
\[V^G \to Z^G\]
is also surjective.

\newcommand{\av}{\mrm{Av}}

Indeed, let $z \in Z^G$. Let $v \in V$ be such that $p\prs{v} = z$. Then
\[p\prs{\av_V^G\prs{v}} = \av_Z^G\prs{p\prs{v}} = \av_Z^G\prs{z} = z \text{.}\]
\end{proof}

\chapter{Decomposition to Irreducible Subspaces}

\begin{lemma}
Assume $V$ is finite-dimensional, there exist $G$-representations $E_1, \ldots, E_n$ such that
\[V \cong \bigoplus_{i \in [n]} E_i\]
as $G$-representations.
\end{lemma}

\begin{proof}
If $V = 0$, it's the empty sum. Then argue by induction on $\dim V$ using \ref{theorem:maschke}.
\end{proof}

We want to understand in which sense the above decomposition is unique.

\begin{lemma}[Schur] \label{lemma:schur}
Let $E,F$ be irreducible $G$-representations. Let $T \colon E \to F$ a $G$-morphism. Then either $T = 0$ or $T$ is an isomorphism.

In particular, if $E,F$ are not isomorphic then
$\hom_G\prs{E,F} = 0$.
\end{lemma}

\begin{proof}
Let $T \colon E \to F$ be a non-zero $G$-morphism.

Consider $\ker\prs{T} \leq E$ which is a $G$-subrepresentation. Therefore since $E$ is irreducible $\ker\prs{T} = 0$ or $\ker\prs{T} = E$, the latter of which implies $T = 0$ in contradiction.
Hence $\ker\prs{T} = 0$ so $T$ is injective.

Consider $\im\prs{T} \leq F$, by irreducibility $\im\prs{T} = 0$ or $\im\prs{T} = F$ the first of which is a contradiction. Hence $\im\prs{T} = F$ so $T$ is surjective.

$T$ is then bijective, hence an isomorphism.
\end{proof}

\begin{claim}
Assume $V$ is finite-dimensional with decompositions as sums of irreducible $G$-representations.
\begin{align*}
V &\cong \bigoplus_{i \in [r]} E_i \\
V &\cong \bigoplus_{j \in [s]} F_j \text{.}
\end{align*}
Then for an irreducible $G$-representation $E$, the number of $i \in [r]$ for which $E_i$ is isomorphic to $E$ i equal to the number of $j \in [s]$ for which $F_j$ is isomorphic to $E$, both of which are equal to
\[\frac{\dim \hom_G\prs{V,E}}{\dim \hom_G\prs{E,E}} \text{.}\]
\end{claim}

\begin{definition}[$G$-endomorphism]
Denote $\endo_G\prs{E} = \hom_G\prs{E,E}$ and call an element of $\endo_G\prs{E}$ a \stress{$G$-endomorphism}.
\end{definition}

\begin{proof}
Denote $d \ceq \dim \endo_G\prs{E} \in \ZZ_{\geq 1}$.
For an irreducible $G$-representation $F$ we have $\dim \hom_G\prs{F,E}$ is zero if $F$ is not isomorphic to $E$ by \ref{lemma:schur}, and is $d$ is $F$ is isomorphic to $E$.
In the latter case there are isomorphisms $F \cong E$ which induce isomorphisms
\[\hom_G\prs{F,E} \cong \hom_G\prs{E,E}\]
defined via pre-composition.

Now
\begin{align*}
\dim \hom_G\prs{V,E} &= \dim \hom_G \prs{\bigoplus_{i \in [r]} E_i, E}
\\&=
\dim\prs{\bigoplus_{i \in [r]} \hom_G\prs{E_i, E}}
\\&=
\dim \sum_{i \in [r]} \hom_G\prs{E_i, E}
\\&=
d \cdot \abs{\set{i \in [r]}{E_i \cong E}}
\end{align*}
where all equalities before the last are due to isomorphisms of the vector spaces.
So
\[\abs{\set{i \in [r]}{E_i \cong E}} = \frac{\dim \hom_G\prs{V,E}}{d} \text{,}\]
as required.
\end{proof}

\begin{definition}
Assume $V$ is finite-dimensional and let $E$ an irreducible $G$-representation. The \stress{multiplicity of $E$ in $V$} is
\[\brs{V:E} \ceq \abs{\set{i \in [r]}{E_i \cong E}}\]
where $V \cong \bigoplus_{i \in [r]} E_i$ is a decomposition to irreducible representations and where $\brs{V:E}$ is well-defined by the previous claim.
\end{definition}

\begin{lemma}[Schur (2)] \label{lemma:schur2}
Assume $k = \bar{k}$ and let $E$ an irreducible irreducible $G$-representation. Then $\endo_G\prs{E} = k \cdot \id_E$.

In particular $\dim \endo_G\prs{E} = 1$.
\end{lemma}

\begin{exercise}
Assume $H \in \Grp^\fin$. An irreducible $H$-representation is finite-dimensional.
\end{exercise}

\begin{proof}
Let $T \in \endo_G\prs{E}$. Since $E$ is finite-dimensional and $k = \bar{k}$ there's $\lambda \in k$ such that $T - \lambda \cdot \id_E$ is not invertible. Then by \ref{lemma:schur}
$T - \lambda \cdot \id_E = 0$, so $T = \lambda \cdot \id_E$, as required.
\end{proof}

\begin{corollary}
By \ref{lemma:schur2} and the previous claim, if $k = \bar{k}$ and $E$ one has
\[\brs{V : E} = \dim \hom_G\prs{V,E} \text{.}\]
\end{corollary}

\begin{definition}[The Regular $G$-Set]
Let $G$ with the left $G$-action by multiplication be \stress{the regular $G$-set}.

I.e. the action is
\[a\prs{g, \tilde{g}} = g \tilde{g} \text{.}\]
\end{definition}

\begin{definition}[The Regular $G$-Representation]
\stress{The regular $G$-representation} is the corresponding $k\brs{G}$ of the regular $G$-set.
\end{definition}

\begin{remark}
The regular $G$-representation has a basis $\set{\delta_g}_{g \in G}$ with $g \cdot \delta_{\tilde{g}} = \delta_{g \tilde{g}}$.
\end{remark}

\begin{remark}
From linear algebra we have a natural isomorphism of vector spaces
\[\hom\prs{k\brs{X}, V} \cong \Maps\prs{X,V}\]
where $\Maps\prs{X,V}$ are arbitrary maps.
This is given by
\[T \mapsto \prs{x \mapsto T\prs{\delta_x}} \text{.}\]
If $X$ is a $G$-set the isomorphism induces an isomorphism of vector spaces
\[\hom_G\prs{k\brs{X}, V} \cong \Maps_G\prs{X,V} \text{.}\]
\end{remark}

\begin{fact}[For those who know category theory]
$k\brs{-}$ is in fact left-adjoint to the functor forgetting the linear structure of a $G$-representation.
\end{fact}

\begin{proposition}\label{proposition:multiplicity_formula}
Let $E$ be an irreducible $G$-representation.
Then
\[\brs{k\brs{G} : E} = \frac{\dim E}{\dim \endo_G\prs{E}} \text{.}\]
\end{proposition}

\begin{proof}
Using the observation in the last remark we get
\begin{align*}
\brs{k\brs{G} : E} &=
\frac{\dim \hom_G\prs{k\brs{G}, E}}{\dim \endo_G\prs{E}}
\\&=
\frac{\dim \Maps_G\prs{G,E}}{\dim \endo_G\prs{E}}
\\&=
\frac{\dim E}{\dim \endo_G\prs{E}} \text{.}
\end{align*}
\end{proof}

\begin{corollary}
There are finitely many irreducible $G$-representations up to isomorphism.
If $k\brs{G} = \bigoplus_{i \in [r]} E_i$ and $E$ is irreducible, it is isomorphic to one of the $E_i$ as by the above proposition $\brs{k\brs{G} : E} \neq 0$. 
\end{corollary}

\begin{corollary}\label{corollary:sum_of_irreducible_dimensions}
Assume $k = \bar{k}$. Let $\prs{E_i}_{i \in [r]}$ be a section of the irreducible $G$-representation up to isomorphism.
Then
\[\abs{G} = \sum_{i \in [r]} \prs{\dim E_i}^2 \text{.}\]
\end{corollary}

\begin{proof}
By direct computation
\begin{align*}
\abs{G} &= \dim k\brs{G}
\\&= \sum_{i \in [r]} \brs{k\brs{G} : E_i} \cdot \dim E_i
\\&=
\sum_{i \in [r]} \prs{\dim E_i}^2
\end{align*}
where in the last equality we use \ref{proposition:multiplicity_formula} and \ref{lemma:schur2}.
\end{proof}

\begin{example}
Let $G = S_3$ and assume $\mrm{char}\prs{k} \notin \set{2,3}$ and $k = \bar{k}$.

There are two characters $S_3 \to k^\times$, the trivial one and $\sgn$.
Corresponding to them are two one-dimensional irreducible representations $k_{\mrm{triv}}, k_{\sgn}$.

Using \ref{corollary:sum_of_irreducible_dimensions} we get that
$G = 1^2 + 1^2 + 2^2$
is the only solution, so there's one other representation which is of dimension $2$.
We've seen another representation of $S_3$,
\[W = \set{\pmat{x_1 \\ x_2 \\ x_3}}{x_1 + x_2 + x_3 = 0} \text{,}\]
which is then the required $2$-dimensional representation.
\end{example}

%29/3/2020

\begin{definition}[$k$-Algebra]
A \stress{$k$-algebra} $A \in \Ring$ which is also a $k$-vector space, such that the multiplication map $A \times A \to A$ is $k$-bilinear.
\end{definition}

\begin{remark}
Given a $k$-algebra $A$ we have a ring homomorphism $k \to Z\prs{A}$ given by $c \mapsto c \cdot 1$. It is injective, so we can think of $k$ as sitting inside $A$.

One can therefore think of a $k$-algebra as a ring $A$ equipped with a ring homomorphism $k \to Z\prs{A}$.
\end{remark}

\begin{definition}[Module]
Let $R \in \Ring$. A \stress{(left) $R$-module} is $M \in \Ab$ equipped with a biadditive map
\begin{align*}
R \times M &\to M \\
\prs{r,m} &\mapsto rm
\end{align*}
such that
\begin{align*}
\forall m \in M \colon & 1 \cdot m = m \\
\forall r_1, r_2 \in R \forall m \in M \colon & \prs{r_1 r_2} \cdot m = r_1 \prs{r_2 m} \text{.}
\end{align*}
\end{definition}

\begin{notation}
Denote by $R-\Mod$ the category of left $R$-modules.
\end{notation}

\begin{definition}[Morphism of Modules]
Let $R \in \Ring$ and $f \colon M \to N$ a map of sets such that $M,N \in R-\Mod$.

$f$ is \stress{a morphism of $R$-modules} if it's additive such that
\[\forall r \in R \forall m \in M \colon \phi\prs{rm} = r \phi\prs{m} \text{.}\]
\end{definition}

\begin{notation}
Denote the category of $k$-algebras, with morphisms between them being morphisms of the respective modules, by $\Alg_k$.
\end{notation}

\begin{remark}
Let $A \in \Alg_k$.
Any $M \in A-\Mod$ has a structure of a $k$-vector-space by
\[k \rmono Z\prs{A} \subseteq A \acts M \text{.}\]
Morphisms of $A$-modules will be $k$-linear. 

Often do we start with a $k$-vector space $M$ and give it the structure of an $A$-module.
We then assume implicitly that the $k$-vector-space structure on $M$ arising from this module structure is the same as the original one.
\end{remark}



\begin{comment}
\begin{thebibliography}{2}
\bibitem{context} 
Emily Riehl. 
\textit{Category Theory in Context}. 

\bibitem{nlab}
\textit{nLab - Online Wiki with Focus on Category Theory and Homotopy Theory}.\\
\href{https://ncatlab.org/nlab/show/HomePage}{https://ncatlab.org/nlab/show/HomePage}
\end{thebibliography}
\end{comment}

\backmatter
\end{document}